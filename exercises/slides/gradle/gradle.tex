\documentclass{beamer}
\usepackage{basileabeam}
\usepackage{listings}
\usepackage{color}
\usepackage{ngerman}
\usepackage{tabularx}
\usepackage{tikz}
\usepackage{dirtree}
\usepackage[normalem]{ulem}


%notes
%\pgfpagesuselayout{2 on 1}[a4paper,border shrink=5mm]
% \setbeamertemplate{note page}[plain]
% \setbeameroption{show notes on second screen=bottom}

\title              {Gradle}

\author             {Silvan Heller \textless silvan.heller@unibas.ch\textgreater}
\email              {}
\institute          {Department Mathematik \& Informatik, Universität Basel}

\date               {HS17-- Software Engineering}


\ulogo        		{Template/header}
\ulistelement    	{Template/listelement}

\graphicspath{{Figures/}}
\definecolor{groovyblue}{HTML}{0000A0}
\definecolor{groovygreen}{HTML}{008000}
\definecolor{darkgray}{rgb}{.4,.4,.4}
 
\lstdefinelanguage{Groovy}[]{Java}{
  keywordstyle=\color{groovyblue}\bfseries,
  stringstyle=\color{groovygreen}\ttfamily,
  keywords=[3]{each, findAll, groupBy, collect, inject, eachWithIndex},
  morekeywords={def, as, in, use},
  moredelim=[is][\textcolor{darkgray}]{\%\%}{\%\%},
  moredelim=[il][\textcolor{darkgray}]{§§}
}
\lstset{language=Groovy,showstringspaces=false}

\begin{document}
\begin{frame}[t,plain]
\titlepage
\end{frame}
\begin{frame}{Warum brauchen wir ein Build System?}
	\begin{itemize}
		\item \texttt{javac -cp libs/junit-4.5.jar:libs/slf4j-2.1.jar *.java \&\& jar cmf my\_manifest project *.class}
		\item 'Manually add our 50 dependencies by hand via the user Interface of your IDE'
		\item 'Before executing the tests, you need to manually run scripts/setup-test.sh' 
	\end{itemize}
\end{frame}

\begin{frame}{Manuelle Builds: Schwächen}
	\begin{itemize}
		\item Unübersichtlich
		\item Systempfadabhängig
		\item \texttt{javac} hat eine unübersichtliche Anzahl an Optionen
		\item Wer ruft den Build Befehl auf? 
		\item Manuelles Verwalten von verwendeten Libraries
	\end{itemize}
		\begin{center}
			\begin{large}
				Die Zeit des Entwicklers ist eine teure Ressource
			\end{large}
		\end{center}
\end{frame}

\begin{frame}{Warum nicht einfach Eclipse oder IntelliJ?}
	Eclipse kann
	\begin{itemize}
		\item Automatisch jars erstellen
	\end{itemize}
	IDEs können meistens nicht (automatisch)
	\begin{itemize}
		\item Libraries verwalten
		\item Optionen von \texttt{javac} abstrahieren
		\item Von überall aufgerufen werden (ssh auf remote server)
		\item Dependencies über mehrere Projekte cachen
	\end{itemize}
\end{frame}

\begin{frame}{Die Lösung}
	\includegraphics[width=\linewidth]{logo_gradle.png}
\end{frame}

\begin{frame}{Gradle ist relevant!}
	\begin{minipage}{0.49\linewidth}
		\includegraphics[width=\linewidth]{logo-linkedin.png}
	\end{minipage}
	\begin{minipage}{0.49\linewidth}
		\includegraphics[width=\linewidth]{logo-netflix.png}
	\end{minipage}

	\begin{minipage}{0.49\linewidth}
		\includegraphics[width=\linewidth]{logo-adobe.png}
	\end{minipage}
	\begin{minipage}{0.49\linewidth}
		\includegraphics[width=\linewidth]{logo-google.png}
	\end{minipage}

	\hspace*{2cm}\begin{minipage}{0.49\linewidth}
		\includegraphics[width=\linewidth]{paypal_logo.jpg}
	\end{minipage}
\end{frame}
\begin{frame}{Gradle: Aufbau}
	\begin{itemize}
		\item Gradle besteht aus zwei Teilen: \textbf{Projekten} und \textbf{Tasks}
	\end{itemize}
	\begin{description}
		\item[Projekt] Sammlung von Aufgaben
		\item[Task] Eine spezifische Aufgabe. Etwa das kompilieren einer Java Klasse 
	\end{description}
	\begin{itemize}
		\item Gradle speichert alle Informationen in einer \textit{build.gradle} Datei 
	\end{itemize}
\end{frame}

\begin{frame}{Gradle: Ordnerstruktur}
	Vorgeschlagene Ordnerstruktur
	\dirtree{%
	.1 src.
	.2 main.
	.3 resources.
	.3 java.
	.4 hello.
	.5 SomeJavaFile.java.
	.2 test.
	.3 java.
	.4 hello.
	.5 SomeJavaFileTest.java.
	.3 resources.
	}
\end{frame}

\begin{frame}[fragile]{Gradle: Dependencies}
	\begin{itemize}
		\item Größere (Java) Projekte benötigen Libraries
		\item Beispiele: HTTP, JSON, Logging, Testing, Math
		\item[$\rightarrow$] Zentrales Repository mit verschiedenen Versionen der Library
	\end{itemize}
\begin{lstlisting}
repositories {
    mavenCentral()
}
dependencies {
    compile "joda-time:joda-time:2.2"
    testCompile "junit:junit:4.12"
    compile fileTree(dir: 'lib', include: ['*.jar'])  
}
\end{lstlisting}
\begin{itemize}
	\item \texttt{compile} und \texttt{testCompile} ermöglichen testspezifische Abhängigkeiten
	\item Es können auch jars manuell eingebunden werden
\end{itemize}
\end{frame}

\begin{frame}{Gradle: Wichtige Befehle (Standard)}
	\begin{itemize}
		\item Projekt bauen
		\begin{itemize}
			\item \texttt{gradle build}
		\end{itemize}
		\item Tests ausführen
		\begin{itemize}
			\item \texttt{gradle test}
		\end{itemize}
		\item Projekt bauen und ausführen
		\begin{itemize}
			\item \texttt{gradle run}
		\end{itemize}
	\end{itemize}
\end{frame}

\begin{frame}{Gradle in Ganttproject}
	\begin{itemize}
		\item \sout{Vorgeschlagene Ordnerstruktur}
		\item \sout{Dependency Management}
		\item \sout{Gradle Wrapper}
		\item Mehrere Subprojekte, jedes hat ein eigenes build.gradle file
		\item Gradle wird zum builden verwendet und nicht / nur halb fürs dependency management
		\item 'gradle runApp' zum starten
	\end{itemize}
\end{frame}
\end{document}
